\documentclass{article}

\usepackage{hyperref}
\usepackage{setspace}

\title{Installing Arch Linux in a Nutshell}
\date{February 2018}
\author{Sagindyk Urazayev}
\onehalfspacing
\begin{document}
\maketitle
\newpage
\tableofcontents
\newpage
\section{Pre-Installation}

Open the Arch Linux Official Website and download the latest .ISO version of the system.\\

Later on just take a flash drive and burn it with rufus on Windows or with any other tool on Linux. \\

\section{Installation}

\subsection{BIOS and Flash Drive}

\subsubsection{BIOS Configuration}

This part is quite easy. The only thing that you should do is to tell your BIOS to boot from Flash Drives instead of bootloader and other drives.\\
This process is different for every BIOS Version and motherboard, however, this is really easy as all the entries in the BIOS menu are self-explanatory.\\

\subsubsection{Flash Drive}

Just plug in your flash drive and power on/reboot your device.\\

\subsection{Booting from Flash Drive}

Now you will see a systemd prompt asking to boot. Choose the boot option that includes Arch Linux in it.\\

\subsection{Root on archios}

Now you will be throwed at the promtpt that looks like this.\\

\begin{center}
  root@archios \#\\
\end{center}

Everything is going very well. Now, this guide includes Arch Linux installation without dual boot. So suppose that Arch Linux is the only Operating System on your device. If you want to go with dual boot with Windows or other OS. Refer to another guide or do it at your own risk.\\

\subsubsection{Disk Partitioning}

Okay, now we need to partition our hard/state drive. How do we do that?\\

Actually, there are many ways to to it, however I prefer to use cfdisk. It has pseudo-GUI, which is easier to work and operate with. Before partitioning, we should find the name of your drive. For that, in your command line type\\

\begin{center}
  fdisk -l\\
  \end{center}

You will see a list of available storages. Ignore everything that ends with: boot, root, include, etc. What you are interested in is the biggest available drive. For example, my laptop has integrated SD Card 27GB and it is called /dev/mmcblk0.\\

For the sake of simplicity, let's use notation /dev/sda for the name of the drive.\\

Okay, good. Now, launch cfdisk by typing\\

\begin{center}
  cfdisk -z /dev/sda\\
\end{center}

You will enter pseudo-GUI interface. Select \textbf{gpt} and now you will see your drive's partitions. Next, we need to create 3 partitions: EFI System(/dev/sda1), Swap memory(/dev/sda2) and Root Directory (x86\_64)(/dev/sda3) . I don't create home partitions, as I just don't like it.\\

By the rules, your swap should be 2xRAM and EFI System 512MB. Everything else can be given to Root Directory. Now, exit cfdisk by choosing QUIT.\\

\subsubsection{Formatting partitions}

We need to format partitions appropriately and accordignly.\\

Your EFI System should be FAT32 formatted, because we use systemd-boot. To do that, type

\begin{center}
mkfs.fat -F32 /dev/sda1\\
\end{center}

Your swap partition should be formatted as swap memory (obviously)

\begin{center}
  mkswap /dev/sda2\\
  swapon /dev/sda2\\
  \end{center}

Your main directory should be formatted as ext4, so to use all your files further on

\begin{center}
  mkfs.ext4 /dev/sda3\\
  \end{center}

Very good, our partitions are properly formatted and we can go on.

\subsection{Mounting}

Now, we should mount our partitions. Firstly, we should mount our root partition

\begin{center}
  mount /dev/sda3 /mnt
\end{center}

Then, we should mount the EFI System, so the system would know where is the boot point. We also need to create boot directory on our own.

\begin{center}
  mkdir /mnt/boot\\
  mount /dev/sda1 /mnt/boot\\
  \end{center}

Awesome, everything is mounted, now we can start installing the system!\\
v
\subsection{Installing the actual OS}

\subsubsection{Finding fast mirrors}

This part is quite easy. However, when I was installing my system, I had a problem with downloading speed, it was something like 20-50KB/s. Awful. To fix that and get the maximum download speed, do the following:

\begin{center}
  cp /etc/pacman.d/mirrorlist /etc/pacman.d/mirrorlist.backup\\
  sed -i 's/\^\#Server/Server/' /etc/pacman.d/mirrorlist.backup\\
  rankmirrors -n 6 /etc/pacman.d/mirrorlist.backup $>$ /etc/pacman.d/mirrorlist\\
  \end{center}

Now you will be connected to the fastest mirrors in your location. Hurray!\\

\subsubsection{Pacstrap}

Now we will install the base system. Here, we will intall base and base-devel packages, because
it will get us enough packages to start using Arch Linux.\\

\begin{center}
  pacstrap /mnt base base-devel\\
  \end{center}

\subsubsection{Genfstab}

Now, the system is installed on the device and we need to tell our OS and Bootloader about the partitions of our disk. To do so, perform:

\begin{center}
  genfstab -U /mnt/etc/fstab
  \end{center}

\end{document}
