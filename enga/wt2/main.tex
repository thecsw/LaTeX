\documentclass[a4paper,12pt]{article}
\usepackage[utf8]{inputenc}
\usepackage{setspace}
\usepackage{url}
\usepackage{hyperref}

\doublespacing

\title{English A Language \& Literature. Written Task 2}
\date{}
\author{}

\begin{document}

% -------------- COVER PAGE STARTS HERE -----------------
\pagenumbering{gobble}
\maketitle
\begin{center}
  If the text had been written in a different time or place or language or for a different audience, how and why might it differ?
\end{center}
\begin{flushleft}
  \begin{figure}
    Session: May 2018\\
    %School number: 007055\\
    %Student number: 0004\\
    Word count: WORK IN PROGRESS\\
    \end{figure}
\end{flushleft}
\newpage
% -------------- COVER PAGE ENDS HERE -----------------
\pagenumbering{arabic}

What if I tell you that all pictures in you house have a hidden camera in it, would you believe me? What if I tell you that you are being watched all the time without you knowing, would you believe me?  What if I tell you that all the lies you know are actually true and are called "alternative facts", would you believe me? I thought not. It is not a truth you would want to know. Unfortunately, it is happening rigth now. Nearly 70 years later, George Orwell has predicted our time with his classic dystopian novel - "1984"\\

1984 is a beautifully written novel by a true genius of his time - Eric Arthur Blair or more known to the public as George Orwell. In his work, Orwelll added a lot of new topic to discussion: physical control, the dangers of totalitarianism, control of information and history, and so much more. During his life, George Orwell was an English novelist, essayist, journalist and critic. He had experience in military journalism, he has seen the horrors of war and poverty, thanks to that, with his pen, he was able to transform the horrors of total contol and the fear of the future on paper. It is fascinating how during his time, in the late 1940s and early 1950s the novel made a such big impact because the idea and the genre was new. It was simultaneously interesting and terrifying. In the TIME article of February 4, 1946 \cite{time}, it can be seen how people of that time were terrified of what the future might look like. \\

\newpage
\bibliographystyle{ieeetr}
\bibliography{main}
\end{document}
