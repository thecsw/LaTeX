\documentclass[a4paper,12pt]{article}
\usepackage[utf8]{inputenc}
\usepackage{setspace}
\usepackage{url}
\usepackage{hyperref}

\doublespacing

\title{English A Language \& Literature. Written Task 2}
\date{}
\author{}

\begin{document}

% -------------- COVER PAGE STARTS HERE -----------------
\pagenumbering{gobble}
\maketitle
\begin{center}
If the text had been written in a different time or place or language or for a different audience, how and why might it differ?
\end{center}
\begin{flushleft}
\begin{figure}
Session: May 2018\\
%School number: 007055\\
%Student number: 0004\\
Word count: WORK IN PROGRESS\\
\end{figure}
\end{flushleft}
\newpage
% -------------- COVER PAGE ENDS HERE -----------------


\pagenumbering{arabic}

What if I tell you that all pictures in you house have a hidden camera in it, would you believe me? 1What if I tell you that you are being watched all the time without you knowing, would you believe me?  What if I tell you that all the lies you know are actually true and are called ``alternative facts'', would you believe me? I thought not. It is not a truth you would want to know. Unfortunately, it is happening rigth now. Nearly 70 years later, George Orwell has predicted our time with his classic dystopian novel - ``1984''. Here, an important question arises - ``If the text had been written in a different time or place or language or for a different audience, how and why might it differ?'' \\

Firstly, in order to talk about how the novel might differ in another timeline, we should understand the difference between the year ``1984'' was published - 1949 and the year that we live in right now - 2018. 1949 year was quite a rough year for the whole world. World War II has just ended. Whole Europe was in ruins. The relationships between the world countries has never been more intense. This unfortunate turn of events and cicumstances have led George Orwell to write his epic novel ``1984''. In his work, Arthur wanted to show the public his concerns and the horrors of the totalitarian structure of governments and total control with propaganda. For his time, George Orwell was in the few first who used the genre of dystopian futures. In 1984" there is a device called ``telescreen'', which was a form of a TV but with a 24/7 camera in it that spied on the people near it. His ideas were fresh and unexplored, it was new. What about now?\\

The technology has advanced a lot during the last 69 years. Now, we have TVs in every house, cellular phones became widely available and an absolutely new technology was developed and implemented - Internet. Internet has changed the way we interact with devices, with people, it has changed the way we communicate and the way we live. It has chnaged our lives so drastically in a such short amount of time. Even George Orwell did not predict the technology. The idea of Internet, of interconnected devices around the world was not even thought about in the year 1949. I mention this, because if ``1984'' was published in our time, it would be sinful to not mention and use the Internet as the means of propaganda, instead of propaganda in the form of posters like in ``1984''.\\


%\newpage
%\bibliographystyle{ieeetr}
%\bibliography{main}
\end{document}
