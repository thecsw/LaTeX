\documentclass{article}
\usepackage[utf8]{inputenc}
\usepackage{setspace}

\title{The Diary of Syme. Rational}
\author{Sagindyk Urazayev}
\date{January 2018}
\doublespacing
\begin{document}

\maketitle
\newpage

After reading George Orwell’s dystopian novel “1984”, I was going through a range of emotions and thoughts about the themes, motifs and ideas. This work became one of my all-time best dystopian novels.\\

During the reading of the book, we encounter a character named Syme, a language expert and Winston’s co-worker. From the conversation between Win- ston and Syme, we learn that he is currently working on the new edition of the Newspeak dictionary, the official language of Oceania. Nevertheless, what differ- entiates him from everybody else, he sees a bigger scope in a way that he knows why he is doing his job and what the party is trying to accomplish by it - by limiting the vocabulary’s size, limit the way people think and ban the thoughts that are against the party and Big Brother. \\


I have decided to write a diary of Syme as I felt that I was missing more information and story development for him, to understand why he vanished. Maybe because of the diary he kept? I considered diary as the best format to express his inner thoughts and suspicions. This will unveil Syme’s attitude towards the Party. In the diary, he will describe the routing of his day, illustrate some of the newspeak words, we can see the details of the conversation with Winston from another perspective and get a “fresh look” at the whole story. This means that in the diary we will relive the events from the book, but from another point of view, thus expanding our comprehension of the book by getting more objective view on the plot. In order for my diary entries to be authentic, I have a date on top of each entry, because in comparison with Winston, Syme was more intelligent and was keeping the track of the dates.\\

\end{document}
