\documentclass[a4paper]{IEEEtran}
\usepackage[utf8]{inputenc}
\usepackage{hyperref}
\usepackage{amsmath}

\title{Analysis and Prediction of the Future of the Observable Universe}
\author{Sagindyk Urazayev}
\date{24-12-17}

\begin{document}

\maketitle

\begin{abstract}
  This Physics Investigation work is aimed at researching and demonstrating events that will happen in the far future of the Observable Universe.
  During the investigation, I will analyze the possibilities of the future events and scientifically backing up my claims with formulae. The needed data will be extracted from available atronomical databases. I have written several programs in programming languages to simulate events that will be discussed in the Investigation. For the sake of demonstration, every major event will be accompanied by an illustration that was made by me, using graphics software.\\
  So, what will we miss?
\end{abstract}

\section{Introduction}

Aristotle was an ancient Greek scientist and philosopher. According to Aristotle’s book – “Physica”, an idea was introduced that all matter around us consisted out of four basic elements: Fire, Earth, Air, and Water\cite{aether}. During his time it was also believed that Earth was the center of the Universe and the Universe can exist forever. It took humanity humanity nearly two millenials to find out that Earth is not the Center of the Universe, it is not even in the center of Solar System. We understood that Universe is not eternal, it has its own beginning and its own end. Actually, Universe is commonly defined as the totality of everything that exists, including all physical matter and energy, the planets, stars, galaxies, and the contents of intergalactic space.\cite{universe}\\

The Observable Universe is so big and enormous, its diameter is around 28 billion parsecs or 93 billion light years.\cite{size} By the Albert Einstein's General Relativity Theory we know that nothing can exceed speed of light, thus making the speed of light the limit of maximum speed\cite{speed}. That means that it would take light, the fastest object ever-known to humanity 93 billion years to travel from one end of the Universe to another. That is about the Universe, what about the planets, stars and galaxies within it?\\

For homo-sapiens light is information. Light is the tool that we use to explore, research and investigate everything that surrounds us. We see all the beauty of cosmos thanks to light. All the emitted radiation, including photons in the visible spectra are carrying information about the place, time and poisition of the source. Even if nothing can exceed the speed of light, it still has its limitations. Speed of light is $299792468ms^{-1}$ or $\propto 3 \times 10^8ms^{-1}$\cite{speed}, so it does take time to travel from one point to another. What if the distances are so enormous that it can take light a lot of time to travel? This is called astronomical distances. The distances are so big that SI units, meters are not even used. Instead, parsecs and light years are used. Light years is the distance that light travels in one calendar year and parsec is the distance to a star, which parallax angle equals to one arcsecond. The astronomical distances will be discussed more thoroughly in next parts of the Investigation.\\

Due to big distances, we, humans on Earth are receiving ``outdated information''. If we have a star and its distance to Earth is 1 light year. When the stars blows up or something catastrophical has happened to the star, we would find it out only 1 year after the event. The same principle is applied to all bodies in the Universe. This is simultaneously fascinating and pity. There is a chance that grandious events have happened and we just did not get the news yet. So, this investigation will research the first astronomical events from the point of view of Earth and later from a more objective point of view. The details are presented in next chapters.\\

People would ask, if Universe hosts everything that existed, exists and will exist, can it have an end? Is it possible for Universe to die and stop existing for an indefinite amount of time? Our latest studies show that Universe has its beginning in Big Bang and its possible end in an event called ``Big Freeze'', where no life can sustain itself and all stars and bright objects in the Universe will die and evaporate. More on that later.\\

This investigation will explore the nature of the Observable Universe, its properties and the possible future. To prove the theories and analyze the possibility of future events, I have written programs and simulations in a programming language - Python and all the data will be taken from reliable and available astronomical databases. To demonstrate the events and how they may look from some position's perspective, I will be using Adobe Photoshop to accurately visualize the scenes.\\

\section{Research Question}

The aim of this Investigation is to thoroughly analyze the available astronomical data and then accurately make  

\bibliography{main}
\bibliographystyle{ieeetr}

\end{document}
