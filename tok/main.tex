\documentclass[12pt,a4paper]{article}
\usepackage[utf8]{inputenc}
\usepackage{setspace}
\usepackage{hyperref}

\title{Theory of Knowledge Essay}
\author{}
\date{}
%Iis connected to the duration of the duration of developemnt of tha sicplines. But how muhc is it direction, i will do
\begin{document}

\pagenumbering{gobble}

\maketitle

\doublespacing

\begin{center}

  ``The quality of knowledge produced by an academic discipline is directly proportional to the duration of historical development of that discipline.'' Explore this claim with referenceto two disciplines.

\end{center}

\begin{figure}[b]
  \begin{flushleft}
    Examination session: May 2018\\
    %    School number: 007055\\
    %    Student number: 0004\\
    Word count: 1640\\
  \end{flushleft}
\end{figure}

\newpage
\pagenumbering{arabic}
%\doublespacing
\begin{center}
  \textbf{``The quality of knowledge produced by anacademic discipline is directly proportional to the duration of historical development of that discipline.'' Explore this claim with reference to two disciplines.}
\end{center}

Knowledge has always been, is and will be an inalienable part of human development, both spiritual and scientific. In the grand quest of reaching new heights of knowledge, people have observed, sought and opened previously hidden concepts and by that, pushing knowledge to its own limits. This strategy awarded us with breakthroughs of the greatest importance that allowed us to understand the world we live in more precisely and thoroughly. \\

Before agreeing or disagreeing with the claim in the title, we need to understand the meaning of ``quality of knowledge'' and
``duration of historical development'' in it. I tend to believe that quality should be understood as a scale on how precisely an academic discipline lets us understand the world around us and further be applied in our daily lives. Duration is the time that humanity has spent time on researching and contributing to an academic discipline. To start off, I want to say that I agree that quality and duration of historical development of an academic discipline are connected but I will discuss how much are they directly proportional further in the essay with reference to Mathematics and Physics. These academic disciplines have a long journey of historical development and they have changed our perception of the world more than once. Time is a fundamental and constant variable in our three-dimensional world\cite{time}.  We can perceive time a straight line with no turns left or right, time goes by and everything with it. What effect does it have on the disciplines and our way of perceiving the world? This leads us to our first Knowledge Question - \textit{“How can time improve the quality of knowledge produced by an academic discipline and shape the way we perceive the world around us?”}\\

Science as a whole is very important to us because it is one of the valid ways in which we can understand the mechanism of our world and the meaning of it. In our world, it works in that way that when knowledge produced by an academic discipline is believed to be true and there is evidence that it can be true, and then it should be validated and depending on the outcome considered true. What will happen, when the knowledge produced by an academic discipline and that is considered true is actually false and does not accurately explain the world as it is? I would like to include an example from the early times of humankind, where it has just begun trying to explain the world that surrounds us. Aristotle was an ancient Greek scientist and philosopher. His views on physical science were the foundation of early and medieval physics. According to Aristotle’s book – “Physica”, an idea was introduced that all matter around us consisted out of four basic elements: Fire, Earth, Air, and Water\cite{aether}. He explained that combination of these four elements in different proportions would give us different materials with different properties. Heavier elements like iron and gold proportionally would have a bigger amount of Earth in it than lighter elements would. In addition, there was the fifth element, called “Aether” or “quintessence” . Aristotle believed that this fifth element is weightless and it does not interact or combine with any other elements. With this approach, he explained the Sun, Moon, stars, planets and everything beyond the sky as “crystal spheres” that rotate our Earth. The impact was so big, that the Aristotelian physics were accepted during the Late Antiquity, Early Middle Ages and even Renaissance. \\

Only during the Age of Enlightenment in the 18th century, scholars and scientists found a new way to represent matter around us – chemical elements. Today we know that there are 94 elements\cite{table}, which make matter, that empty space consists of elementary particles and that Earth rotates Sun in such small scale that comparing it with the universe, it is safe to say that we are practically ``nothing''. Now, returning to the time, where we do not know that any matter is composed of 94 chemical elements, but from 4 classical elements, and that Sun, Moon and all other planets rotate around Earth instead of Earth rotating Sun, would we believe in the Aristotelian Physics? Yes, we will. This is because of a phenomenon called - frame of reference. When the knowledge produced by an academic discipline is the only knowledge that we can refer to, then our understanding of the world will be completely shaped by that knowledge. What can we conclude about all of this? We can conclude that quality is a cause of rather long and slow process that requires time, trials and errors. In the example above, it can be seen how Physics’ understanding of the world was shifting and changing drastically over time. Quality of knowledge produced by it was improving as people were spending time for the sake of progress. Eventually, along the way, many discussions and debates were conducted to find the right path to follow. This leads to our second Knowledge Question - \textit{``Can debates and disagreements in a discipline help to speed up the process of knowledge production and ensure that its quality does increase over time?''}\\

Mathematics and Physics have some disagreements or inconsistencies and the biggest amount of them can be found in Physics. This happens because of a different interpretation of the world and thus the creation of different theories, which solutions are distinct and most of the time even incompatible. Physics can be divided into three parts: Classical Physics, Relativity, and Quantum Mechanics\cite{bop}. All three of them are considered to be an essential part of Physics, therefore true and logical. Nevertheless, everything that is not Classical Physics: Relativity and Quantum Mechanics, they do not converge with each other. Relativity heavily relies on the concepts and theories of gravity, whereas in Quantum Mechanics gravity is not even well-defined. This creates big debates between these two areas. Physicists all over the world are trying to invent “The Theory of Everything”\cite{toe}, which will finally find a bridge between Relativity and Quantum Mechanics, thus making Physics more consistent. Right now, there are good hopes that the discovery of the theory will happen soon, because the current debates are scattering all over the world, thus attracting new and young minds that have a great enthusiasm towards undefined concepts and unsolved problems. We can call this ``hype'' because debates are quite controversial events, thus making “lots of noise” in the public, which eventually will have some attention from outside. How is this connected to the claim in the title? From our nature, people are more interested in proving or disproving statements than creating one. \\

Eventually, when researchers, scientists or any person encounters a claim by knowing that it is wrong, they will try their best to find a true answer. This is what effect debates and discussions have on the development of disciplines. People are becoming interested and engaged with the facing problem, further on they will do everything that they can to find the right answer. This behavior speeds up the process of development of a discipline, as more and more individuals are working on it. Worth noticing that when a large number of people are working on a similar problem, it ensures that the knowledge is correct and quality is increasing. This happened due to what is called – “Peer Review”\cite{tok}. When one’s knowledge or claim is to be shared, firstly it needs to be checked by other scholars for allowing or disallowing the claim to be spread out. We can even call this as “Quality Control”. More there is a heat going on in the field, the more enthusiasts and great minds it does attract. \\

I would like to prove my statements by a second example, by my own experience. My Mathematics exploration topic is called “Convergence of Divergent series”\cite{cods}. In my exploration, I talk about finding finite values for obvious infinite series. For example, we have a series that looks like this: 1+2+3+4+5… When you ask your friend, teacher or anyone and ask them “What is the sum of these elements when I go to infinity?” Most of the time people would tell you that with each new term the value would explode into a bigger value, thus going straight to the infinity. My work was mostly concentrated on disproving the obvious answer and finding a more elegant solution. Actually, the correct value is -1/12. Sounds really strange and unintuitive. Sum of infinite increasing positive integers equals to a small and negative fraction. The new solution is based on zeta function\cite{zeta}, which is a central topic of “Riemann Hypothesis”\cite{hypot}. This topic should be considered controversial, as many mathematicians agree with the statement and many others oppose it. Fortunately, this led to a massive discussion about whether the solution should be valid and it has had a lot of attention. People started to work in this sphere more and more active, this behavior led to new discoveries in the field of infinite mathematics. Later on, when the famous hypothesis will be proved, it will unveil the great secrets of primes number and a countless number of other theories that rely on it. All the works are being “peer reviewed” to ensure that new solutions are qualitative.\\

In the conclusion, it is safe to say that the quality of knowledge produced by an academic discipline is proportional to the duration of historical development of that discipline. We cannot say that it is directly proportional, because there are processes, which can improve quality or speed up the development of given discipline. We saw how that could happen in Physics, where internal inconsistencies of the discipline can encourage new researchers to start working on solutions of problems, thus making the knowledge produced by the discipline more reliable and trust-worthy. \\

\singlespacing
\newpage
\bibliography{main}
\bibliographystyle{ieeetr}

\end{document}
