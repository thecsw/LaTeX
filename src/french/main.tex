\documentclass[a4paper, 12pt]{article}
\usepackage[utf8]{inputenc}

\usepackage{setspace}
\usepackage{hyperref}
\usepackage{url}
\usepackage[french]{babel}

\title{Style de vie des adolescents français et kazakhs}
\author{}
\date{}

\doublespacing

\begin{document}
\maketitle
\pagenumbering{arabic}

Aujourd'hui, les adolescents représentent environ 25\% \cite{teenagers} de la population mondiale, 25.7\% au Kazakhstan et 18.6\% au France. Il peut sembler que tous les adolescents sont pareils, mais les adolescents dans chaque pays sont uniques. Je viens du Kazakhstan et je suis un adolescent. Je parlerai des adolescents au Kazakhstan et en France. \\

Premièrement, les adolescents kazakhs et français diffèrent beaucoup.
C'est à cause de la culture et du pays.
Par exemple, les adolescents français vont habituellement à des événements social: humanitaire ou écologique.
Puis-que les adolescents français et les Européens aiment aider les gens.
Aussi, les adolescent françai and kazakhs aiment lire des livres. \cite{hobbies}
Ils aiment lire des livres classiques et romantique.\\

Ce qui m'a surpris à propos des adolescents français, c'est que en France, les adolescents quittent la maison de leurs parents lorsqu'ils ont 16 ou 18 ans, mais au Kazakhstan, les adolescents vivent plus longtemps avec les parents.
En France, les adolescents économisent de l'argent de poche pour acheter des choses.
Cela rend les adolescents français plus organisés et responsables.
Mais au Kazakhstan, les adolescents kazakhs demandent aux parents de leur donner de l'argent au lieu de le economiser.
Puis-que les adolescents français quittent leurs parents plus tôt, ils sont plus responsables avec l'argent.\\

Les adolescents français seront surpris que les Kazakhs mangent aussi des chevaux. \cite{horse} Les pays européens ne mangent pas de viande de cheval, sauf la France. \cite{france}\\

Je pense ces similitudes et différences culturelles existent parce que ces adolescents ont grandi dans le pays et cultures differents.
Ces adolescents ont differents vues du Monde.
Les Kazakhs étaient des nomades et faisaient partie de l'Union Soviétique.
Ils étaient moins démocratiques et libéraux que les Français.
Au cours de l'histoire, nous vivions dans des endroits différents et c'ets pourquoi la familles et les traditions sont tres importantes pour nous.
Parce que les Kazakhs étaient des nomades et faisaient partie de l'Union Soviétique.
Ils étaient moins démocratiques et libéraux que les Français.
En revanche, les francais, développer la démocratie et ils ont un point de vue plus libéral.\\

% ------------------------- LEGACY STUFF ----------------------------------

%L'histoire de la nation ou des peuple impose certaines mots dans le disctionaire de la langue.
%Par example, en France les adolescents de 16 et de 18 ans quittent leurs maisons paternelles.
%Puis-qu'ils se debrouillent seul en faisont de differentes activites dans leur vocabulaire on trouve de mots:
%economiser, gagner son argent, trouver du travail, etre responsable, survivre.\\

Nombre des mots: 335

\bibliographystyle{ieeetr}
\bibliography{main}

\end{document}
