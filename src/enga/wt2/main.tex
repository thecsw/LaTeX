\documentclass[a4paper,12pt]{article}
\usepackage[utf8]{inputenc}
\usepackage{setspace}
\usepackage{url}
\usepackage{hyperref}

\doublespacing

\title{English A Language \& Literature. Written Task 2}
\date{}
\author{}

\begin{document}

% -------------- COVER PAGE STARTS HERE -----------------
\pagenumbering{gobble}
\maketitle
\begin{center}
If the text had been written in a different time or place or language or for a different audience, how and why might it differ?
\end{center}
\begin{flushleft}
\begin{figure}[b]
Session: May 2018\\
%School number: 007055\\
%Student number: 0004\\
Word count: 999\\
\end{figure}
\end{flushleft}
%\newpage
% -------------- COVER PAGE ENDS HERE -----------------
\pagenumbering{arabic}
% -------------------- OUTLINE ------------------------------
\singlespacing
\textit{Title of the text for analysis:} 1984

\textit{Part of the course to which the task refers:} Part 4: Literature - Critical Study 

\textit{My critical response will:}

\begin{itemize}

\item Explore the importance of the setting and propaganda in 1984. Why the ideas in 1984 would still be relevant today.

  \item Elucidate how governments can have a total control over population due to propaganda.

  \item Conclude with the changes that would have been made if 1984 was written in our days.
    
  \end{itemize}
\doublespacing
\section*{Written Task 2}

% -------------------- HOOK ------------------------------
What if I tell you that all pictures in your house have a hidden camera in it, would you believe me?
What if I tell you that you are being watched all the time without you knowing, would you believe me?
What if I tell you that all the lies you know are actually true and are called ``alternative facts'', would you believe me?
I think not.
It is not the truth you would want to know.
Unfortunately, it is happening right now. Nearly 70 years later, George Orwell has predicted our time with his classic dystopian novel - ``1984''.
Here, an important question arises - ``If the text had been written in a different time or place or language or for a different audience, how and why might it differ?'' \\


% -------------------- INTRODUCTION ------------------------------

First of all, in order to talk about how the novel might diverge in another timeline, we should understand the difference between the year ``1984'' was published - 1949 and the year that we live in right now - 2018. The year 1949 was quite a harsh year for the whole world. World War II just ended. Whole Europe was in ruins. The relationships between the world countries have never been more intense. This disastrous turn of events and circumstances have led George Orwell to write his epic novel ``1984''. In his work, Orwell wanted to show the public his concerns and the horrors of the totalitarian structure of governments and total control with propaganda. For his time, George Orwell was among the few first who used the genre of dystopian futures.\\ 

% -------------------- SETTING ---------------------
Let's talk about the setting in 1984. From the very first pages of the novel, we can feel the dark atmosphere and setting of the story. Even the very first sentence gives out the nature of the story - ``It was a bright cold day in April, and the clocks were striking thirteen.''. It is not a coincidence that George Orwell chose the word ``cold'', the cold weather never meant a good day, it always meant something sinister and dark. Next, Orwell stars to describe the environment of 1984 - ``The hallway smelt of boiled cabbage and old rag mats.'', readers would immediately picture a mental image of an old building with a horrible smell. George Orwell did not stop at old rag mats and boiled cabbage. In 1984 he visioned London government structure as colossal white pyramids that all other buildings were nearly nothing in comparison with them. Orwell created these constructions to show how the government of Oceania was above the citizens, civilians feared the architecture. He showed us the horrors of totalitarian and total control with fear. Orwell valued the setting very much because he wanted his readers to imagine his new world, his new universe that he had created. How his visions of settings can change now?\\

% ---------------------- NEW SETTING --------------------------
During the last 69 years, people have developed a new taste for style, for buildings, for the style of life. I do not think that in the modern version of 1984 flats will be filled with a smell of rotten cabbage and old rags on the floor. I think the new version would have a shiny, polished and modern building that suit the needs and likes of modern man. Because the year is 2018, people would have smart houses that would execute commands at people's will. For example, instead of manually turning on the TV, everything can be done remotely and automatically. However, there is something else than setting and it is far more powerful at influencing people.\\

% --------------------- PROPAGANDA ---------------------------
The key element of 1984 is the horrifying version on how governments and individuals can control people's minds with the use of propaganda and other techniques. For example, from the first chapter we get the description of one of the many propaganda posters in 1984 - ``At one end of it a coloured poster, too large for indoor display, had been tacked to the wall. It depicted simply an enormous face, more than a metre wide: the face of a man of about forty-five, with a heavy black moustache and ruggedly handsome features.''. It was Big Brother. Though it is being debated, I think that Big Brother never existed. He was the figure that was manufactured by the Oceania government. You might ask why? It is being connected with the way we think. Governments are faceless institutions and it can be challenging and difficult for people to believe in something that they do not know how it looks. Big Brother was created as a representation of patriotism and control of Oceania, the figure that citizens would see and praise. George Orwell really thought ahead of his time and we might ask, how would his 1984 look now? How might the setting and propaganda techniques differ?\\

% -------------------- NEW PROPAGANDA ----------------------------
We have something that Orwell and all other people of his time did not have access to and it is the media. The new glorified and never-ending media with new technologies, like phones, smart houses, and the Internet. In comparison with technologies in 1949, now, information can be transmitted almost instantly and without a delay. Instead of posters hanging on the wall, governments of 2018 use social media, like Facebook, Twitter and other means of communication to spread information that they want citizens to hear and believe. Let me tell you, this is even more terrifying than the old way because of the ``higher ups'' wanted to change something fast, they can do it. Previously, on the rise of social media, politics, and Internet were separate institutions. However, lately, social media has helped individuals to become presidents by spreading their word on the world wide web.

% --------------------- CONCLUSION -------------------------------
Thus, from the analysis, we can come to a conclusion that a lot of things have changed from George Orwell's time. New technologies have been developed, new manipulation techniques have been invented and used. Most importantly, the idea of control and manipulation stays the same and that is where George Orwell was right and this is why 1984 is considered a timeless classic.

%\newpage
%\bibliographystyle{ieeetr}
%\bibliography{main}
\end{document}
