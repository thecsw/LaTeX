\documentclass[a4paper, 12pt]{article}
\usepackage[utf8]{inputenc}

\usepackage{setspace}
\usepackage{hyperref}
\usepackage{url}

\title{Style de vie des adolescents français et kazakhs}
\author{}
\date{}

\doublespacing

\begin{document}
\maketitle
\pagenumbering{arabic}

Aujourd'hui, les adolescents représentent environ 25\% \cite{teenagers} de la population mondiale. Il peut sembler que tous les adolescents sont pareils, mais les adolescents dans chaque pays sont uniques. Je viens du Kazakhstan et je suis un adolescent. Je parlerai des adolescents au Kazakhstan et en France. \\

Premièrement, les adolescents kazakhs et français diffèrent beaucoup. C'est à cause de la culture et du pays. Par exemple, les adolescents français vont habituellement à des foires et à des événements humanitaire. Puis-que les adolescents français et les Européens aiment aider les gens. En effet, en France, les adolescents quittent la maison de leurs parents lorsqu'ils ont 16 ou 18 ans, mais au Kazakhstan, les adolescents vivent plus longtemps avec les parents. Aussi, les adolescent françai and kazakhs aiment lire des livres. \cite{hobbies} \\

En France, les adolescents économisent de l'argent de poche pour acheter des choses.
Cela rend les adolescents français plus organisés et responsables.
Mais au Kazakhstan, les adolescents kazakhs demandent aux parents de leur donner de l'argent au lieu de le economiser.
Puis-que les adolescents français quittent leurs parents plus tôt, ils sont plus responsables avec l'argent.\\

Je pense ces similitudes et différences culturelles existent parce que ces adolescents ont grandi dans le pays et cultures differents.
Ces adolescents ont differents vues du Monde.
Les Kazakhs étaient toujours nomades.
Au cours de l'histoire, nous vivions dans des endroits différents et nous avons apprécié nos traditions et respecté nos aînés.
En raison de l'Union Soviétique, les kazakhs sont plus conservatifs.
En revanche, les francais, développer la démocratie et ils ont un point de vue plus libéral.
Ils ont ete tous différents, ils ont des qualités et des risques et nous espérons qu’ils seront tous plus tart de bons citoyens pour leurs pays.\\

L'histoire de la nation ou des peuple impose certaines mots dans le disctionaire de la langue.
Par example, en France les adolescents de 16 et de 18 ans quittent leurs maisons paternelles.
Puis-qu'ils se debrouillent seul en faisont de differentes activites dans leur vocabulaire on trouve de mots:
economiser, gagner son argent, trouver du travail, etre responsable, survivre.\\

Nombre des mots: 336

\bibliographystyle{ieeetr}
\bibliography{main}

\end{document}
