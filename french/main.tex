\documentclass[a4paper, 12pt]{article}
\usepackage[utf8]{inputenc}

\usepackage{setspace}
\usepackage{hyperref}
\usepackage{url}

\title{Style de vie des adolescents français et kazakhs}
\author{}
\date{}

\doublespacing

\begin{document}
\maketitle
\pagenumbering{arabic}
Aujourd'hui, les adolescents représentent environ 25\% \cite{teenagers} de la population mondiale. Il peut sembler que tous les adolescents sont pareils, mais les adolescents dans chaque pays sont uniques. Je viens du Kazakhstan et je suis un adolescent. Je parlerai des adolescents au Kazakhstan et en France. \\

\textbf{Dans quelle mesure l’élève réussit-il à présenter une ou plusieurs différences et/ou similarités
  culturelles entre le sujet culturel choisi dans la ou les cultures associées à la langue cible et ce
même sujet dans sa propre culture ?}\\

Premièrement, les adolescents kazakhs et français diffèrent beaucoup. C'est à cause de la culture et du pays. Par exemple, les adolescents français vont habituellement à des foires et à des événements caritatifs. Parce que les adolescents français et les Européens aiment aider les gens. En effet, en France, les adolescents quittent la maison de leurs parents lorsqu'ils ont 16 ou 18 ans, mais au Kazakhstan, les adolescents vivent plus longtemps avec parents. Aussi, les adolescent françai and kazakhs aiment lire des livres. \cite{hobbies} \\

\textbf{Quel aspect du sujet que vous
avez choisi vous a étonné ?}\\

En France, les adolescents économisent de l'argent de poche pour acheter des choses. Cela rend les adolescents français plus organisés et responsables. Mais au Kazakhstan, les adolescents kazakhs demandent aux parents de leur donner de l'argent au lieu de le sauver. Parce que les adolescents français quittent leurs parents plus tôt, ils sont plus responsables avec l'argent.\\

\textbf{Selon vous, pourquoi ces similarités/différences culturelles existent-elles?}\\

Je pense ces similitudes et différences culturelles existent parce que ces adolescents ont grandi dans le pays et cultures differents. Ces adolescents ont different vues du Monde. Les Kazakhs étaient toujours nomades. Au cours de l'histoire, nous vivions dans des endroits différents et nous avons apprécié nos traditions et respecté nos aînés. En raison de l'Union soviétique, les kazakhs sont plus conservateurs.
En revanche, les francais, développer la démocratie et ils ont un point de vue plus libéral. Ils sont tous différents, ils ont des qualités et des risques et nous espérons qu’ils seront tous plus tart de bons citoyens pour leurs pays.

\bibliographystyle{ieeetr}
\bibliography{main}

\end{document}
